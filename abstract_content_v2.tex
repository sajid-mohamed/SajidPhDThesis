\glsresetall
\hyphenation{ana-lysis}

\begin{sloppypar}
Over the last years, cameras have become an integral component of modern cyber-physical systems due to their versatility, relatively low cost and multi-functionality. Camera sensors form the backbone of modern applications like \glspl{adas}, visual servoing, telerobotics, autonomous systems, electron microscopes, surveillance and augmented reality. \Gls{ibc} systems refer to a class of data-intensive feedback control systems whose feedback is provided by the camera sensor(s). \Gls{ibc} systems have become popular with the advent of efficient image-processing algorithms, low-cost \gls{cmos} cameras with high resolution and embedded multiprocessor computing platforms with high performance. The combination of the camera sensor(s) and image-processing algorithms can detect a rich set of features in an image. These features help to compute the states of the \gls{ibc} system, such as relative position, distance, or depth, and support tracking of the object-of-interest. Modern industrial compute platforms offer high performance by allowing parallel and pipelined execution of tasks on their multiprocessors.
\end{sloppypar}
The challenge, however, is that the image-processing algorithms are compute-intensive and result in an inherent relatively long sensing delay. State-of-the-art design methods do not fully exploit the \gls{ibc} system characteristics and advantages of the multiprocessor platforms for optimising the sensing delay. The sensing delay of an \gls{ibc} system is moreover variable with a significant degree of variation between the best-case and worst-case delay due to application-specific image-processing workload variations and the impact of platform resources. A long variable sensing delay degrades system performance and stability. A tight predictable sensing delay is required to optimise the \gls{ibc} system performance and to guarantee the stability of the \gls{ibc} system. Analytical computation of sensing delay is often pessimistic due to image-dependent workload variations or challenging platform timing analysis. Therefore, this thesis explores techniques to cope with the long variable sensing delay by considering application-specific \gls{ibc} system characteristics and exploiting the benefits of the multiprocessor platforms. Effectively handling the long variable sensing delay helps to optimise \gls{ibc} system performance while guaranteeing \gls{ibc} system stability.

First, this thesis presents the model-driven \gls{spade} flow for \gls{ibc} systems modelling, analysis, design and implementation.  The \gls{spade} flow expressly targets \gls{mpsoc} platforms. The thesis develops the \gls{spade} flow with a focus on the \gls{compsoc} platform. The thesis further presents an adaptation of the \gls{spade} flow for modern industrial platforms – NVIDIA Drive PX2 and NVIDIA AGX Xavier, which are closed-source and difficult to predict. The \gls{spade} flow is validated using the \gls{imacs} framework developed as part of this thesis.
The \gls{spade} flow is explained incrementally in this thesis starting with the platform-specific aspects and later showing how the application-specific aspects are integrated.   

The platform-specific aspects explored in this thesis are application parallelism and pipelining of the control loop. First, we examine the case of application parallelism with no pipelining allowed for the control loop. A \gls{sadf} models the \gls{ibc} system, capturing both the image-workload variations and the parallelism in the image-processing as workload scenarios for a given platform allocation. The contribution of this thesis concerning this case is the relation between dataflow timing analysis and control timing parameters to obtain a tight predictable sensing delay considering implementation constraints. Second, pipelining without parallelising the sensing application is considered. The challenge with pipelining is that the parameters which are relevant for practical implementation – inter-frame dependencies, system nonlinearities and constraints on system variables – are typically not addressed. As a contribution, this thesis presents a \gls{mpc} formulation for pipelined \gls{ibc} systems considering workload variations, inter-frame dependencies, system nonlinearities and constraints on system variables. Third, this thesis considers pipelining and parallelism together. It presents model transformations for modelling, analysis and mapping \gls{ibc} systems using \gls{sadf}. The model transformations allow to relate the dataflow timing analysis to the control timing parameters and to optimise the mapping while considering pipelining and parallelism together. 

The first application-specific characteristic explored in this thesis is the impact of image-workload variations on the \gls{ibc} system performance. The first contribution is the modelling and designing of \gls{ibc} systems by explicitly considering image-workload variations using a scenario-based design approach. The image-workload variations are identified and modelled as a \gls{dtmc}, where each Markov state represents a workload scenario. At runtime, the \gls{ibc} system switches between workload scenarios based on image workload. Having numerous switching workload scenarios results in an unstable system or degrades system performance. This thesis presents a controller synthesis method based on the \gls{mjls} formulation. System scenarios are identified that abstract multiple workload scenarios based on camera frame rate, sensing delay and sampling period. The \gls{dtmc} is then recomputed, considering only the system scenarios, and the controller is synthesised based on the \gls{mjls} formulation. At runtime, the \gls{ibc} system implementation is switching between the system scenarios. It is shown that the presented approach has a better performance compared to the state-of-the-art. This thesis also provides design guidelines on the applicability of state-of-the-art control design methods for given \gls{ibc} system requirements, implementation constraints and system knowledge.

The second application-specific characteristic explored in this thesis is the impact of approximate computing on the \gls{ibc} system performance. Approximate computing trades off accuracy in the signal processing for gains in response time. Approximating the camera \gls{isp} stage helps to drastically reduce the sensing delay at the cost of errors in the sensing processing. The second contribution of this thesis is the approximation-aware design of an \gls{ibc} system. First, the resilience of the given \gls{ibc} system to different approximation choices is analysed. Second, for each approximation choice, the sensing delay is computed, and the error due to approximations is quantified. The approximation-aware controller is then designed, for each approximation choice, by considering the sensing delay and modelling the quantified error due to approximations as sensor noise.  For the analysis and validation, the thesis presents an \gls{imacs} with support for \gls{sil} simulation and \gls{hil} validation. The \gls{imacs} framework models the environment and dynamics using a physics simulation engine, e.g. Webots, CoppeliaSim or Matlab, and interacts with the \gls{ibc} algorithm in the \gls{sil} or \gls{hil} setting. 

In conclusion, this thesis aims to efficiently cope with the long variable sensing delay of \gls{ibc} systems so that engineers can deploy \gls{ibc} systems efficiently in time- and safety-critical domains. The thesis copes with the long variable delay by considering both application-specific characteristics and platform-specific constraints to optimise \gls{ibc} system performance and stability. 
The proposed \gls{spade} flow exploits the application-specific characteristics and platform-specific constraints of \gls{ibc} systems to cope with the long variable sensing delay and optimise the system performance. The \gls{spade} flow explicitly considers image-workload variations, the approximation of \gls{isp}, parallelisation of sensing processing, and pipelining of the control loop.
The \gls{spade} flow is targeted for a predictable and composable \gls{compsoc} platform. This thesis also details how the \gls{spade} flow can be adapted for industrial platforms.
The techniques presented in this thesis achieve substantial control performance improvements compared to the state-of-the-art approaches for a given platform allocation while guaranteeing stability.