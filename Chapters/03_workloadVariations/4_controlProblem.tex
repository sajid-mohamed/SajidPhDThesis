We consider a regulation problem for the \gls{ibc} system. That is, the control objective is to design $u[k]$ such that $y[k] \rightarrow \controlRef$ (a constant reference) as $k \rightarrow \infty$. 
The control objectives can be performance-oriented, control effort/energy-oriented or a combination of both. The control performance quantifies, in essence, how fast the output $y[k]$ reaches the reference $\controlRef$.  The control effort is the amount of energy or power necessary for the controller to perform regulation. 
The control performance and effort are parameters that can be tuned in the cost function for the \gls{lqr} design and \gls{mjls} synthesis using the state and input weights. We evaluate \gls{qoc} of an \gls{ibc} application considering the following metrics: two commonly used control performance metrics - \gls{mse} (explained in Section \ref{sec:ch5_MSE}) and \gls{st} (explained in Section \ref{sec:ch7_QoC_metrics}) and two commonly used metrics to evaluate control effort/energy - \gls{psd} and \gls{mce}. 
 
\noindent\textbf{\Acrfull{psd}:}
The \gls{psd} of a signal describes the power present in the signal as a function of frequency, per unit frequency. It tells us where the average power is distributed as a function of frequency.
We use Welch's overlapped segment averaging spectral estimation method~\cite{welch1967use} to compute \gls{psd} of our control input.
Lower \gls{psd} for the control input signal implies that the energy required is less and hence \gls{qoc} is better.

\noindent\textbf{\Acrfull{mce}:}
We define the maximum control effort as $\max_{k} \norm{u[k]}$.
A lower \gls{mce} means better \gls{qoc}. \gls{mce} can also be used to identify input saturation, if any.