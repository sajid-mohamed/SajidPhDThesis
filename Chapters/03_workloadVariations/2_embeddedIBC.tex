We consider a setting for an \gls{ibc} system as shown in Fig.~\ref{fig:ch3_IBC_block_diagram}. Our sensor is the camera module that captures the image stream.
The image stream is then fed to an embedded platform, e.g., a \gls{mpsoc}, at a fixed frame rate (fps), e.g., 30 fps. The tasks in our \gls{ibc} application - compute-intensive image sensing and processing (\taskS), control computation (\taskC) and actuation (\taskA) -
are then mapped to run on this \gls{mpsoc}.

\subsection{\Gls{lti} systems}~\label{sec:ch3_LTI}
We consider an \gls{lti} system of the following form:
\begin{align}
\dot{x}(t) &= \Acont x(t) + \Bcont u(t), \label{eq:ch3_contsys}\\
y(t) &=  \Ccont x(t),\nonumber
\end{align}
where $x(t) \in \Real^n$ represents the {\it state}, $y(t) \in \Real$ represents the {\it output}  and $u(t) \in \Real$ represents the control {\it input} of the system at any time $t \in \Real_{\geq 0}$. $\Acont$, $\Bcont$ and $\Ccont$ represent the state, input and output matrices of the system, respectively. 

We illustrate our work using the motivating case study of a vision-based lateral control system model explained in Section \ref{sec:lkas_case_Study}. We consider the \gls{lti} model with $v_x=15$$m/s$, and
\\

%\vspace{2em}
\scalebox{0.8}{
  $\Acont = \left[
  \begin{array}{rrrrr}
    -10.06 & -12.99 & 0 & 0 & 0\\
    1.096 & -11.27 & 0 & 0 & 0\\
    -1.000 & -15.00 & 0 & 15 & 0\\
    0 & -1.000 & 0 & 0 & 15\\
    0 & 0 & 0 & 0 & 0\\
  \end{array}
\right],\ \Bcont=\left[
  \begin{array}{c}
    75.47\\
    50.14\\
    0\\
    0\\
    0\\
  \end{array}
\right].$ \\
}
\\[1.5ex]
The fifth system state captures the curvature of the road at the look-ahead distance. The fifth state is required for observability of road curvature. The control input $u(t)$ is the front wheel steering angle $\delta_f$ and the output $y(t)$ is the look-ahead distance $y_L$ leading to $\Ccont = [0\ 0\ 1\ 0\ 0]$.

\subsection{Implementation, control law and control configurations}~\label{sec:ch3_control law}
The discrete-time control implementation we consider is already explained in Section~\ref{sec:ch5_embedded_implementation}.
The control input $u[k]$ is a \emph{state feedback} controller of the following form,
 \begin{align}
		u[k] = \Kgain_{\workloadScenario} z[k] + \Fgain_{\workloadScenario} \controlRef \nonumber
% 		\label{eqn:u}
 \end{align}
where $\Kgain_{\workloadScenario}$ is the state feedback gain and $\Fgain_{\workloadScenario}$ is the feedforward gain both designed for the workload scenario $\workloadScenario$. $\controlRef$ is the reference value for the controller. The approaches we use for designing the gains are explained in Section~\ref{sec:ch3_control_design}.

For each workload scenario $\workloadScenario$, we then define a control configuration $\Configuration_{\workloadScenario}^c$ as a tuple $\Configuration_{\workloadScenario}^c = (h_{\workloadScenario},\tau_{\workloadScenario}, \Kgain_{\workloadScenario}, \Fgain_{\workloadScenario})$.