In this chapter, we consider the \gls{sadf} model of Fig. \ref{fig:ch5_SADF}(a) with the same execution time parameters as our application model. The modelling, mapping and analysis of this \gls{sadf} model has already been explained in detail in Chapter \ref{chap:parallelisation} and is not repeated here. 
In Chapter \ref{chap:parallelisation}, the workload variations are characterised using a PERT distribution~\cite{adyanthaya2014robustness} (see distribution in Fig.~\ref{fig:ch3_workload_IBC}). Using this information, the probability of frequently occurring workload scenarios are characterised. However, information regarding scenario transitions is not captured. This means that any arbitrary order for scenario switching sequences needs to be considered in the language of the \gls{sadf} model.

In the \gls{mjls}-based approach, the workload variations are characterised using a \gls{dtmc} that takes into account the scenario transition probabilities. The states of the \gls{dtmc} model the workload scenarios (see Section~\ref{sec:ch3_MJLS}) and the transitions in the \gls{dtmc} model the scenario transitions. This means that the \gls{dtmc} determines the language of the \gls{sadf} model~\cite{theelen2006scenario}.