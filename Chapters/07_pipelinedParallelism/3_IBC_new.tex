\section{Multiprocessor \texorpdfstring{\gls{ibc}}{IBC} implementation}
\label{sec:ch7_embeddedIBC}
We consider a typical setting for an \gls{ibc} system as shown in Fig.~\ref{fig:ch7_ibc_intro}~(a) having the workload distribution as illustrated in Fig.~\ref{fig:ch7_ibc_intro}~(c). The main sensor is a camera module that captures the image stream. The image stream is then fed to an embedded multiprocessor platform at a fixed \gls{fps}, e.g.\ 60 fps and image arrival period~$\fh$ given by ${\fh=1/60=16.67\text{~ms}}$. 
The tasks include compute-intensive image sensing and processing (\taskS), control computation (\taskC), and actuation (\taskA), which are then mapped to run on a multiprocessor platform.
We illustrate our work using the motivating case study of a vision-based lateral control system model explained in Section~\ref{sec:lkas_case_Study}.
Further, we consider predictable and composable \gls{mpsoc} platform \gls{compsoc} (explained in Section~\ref{sec:compsoc}) for illustrating the \gls{spade} flow for pipelined parallelism. We also adapt the \gls{spade} approach for the industrial platform NVIDIA AGX Xavier (explained in Section~\ref{sec:nvidia_agx}) to demonstrate its applicability in an industrial context.

\subsection{Control system and embedded implementation}
We consider a \gls{lti} feedback control system model for \gls{ibc} given by:
\begin{align}
\label{eq:ch7_contsys}
\dot{x}(t) &= \Acont x(t) +  \Bcont u(t),\\
y(t) &= \Ccont x(t) + \Dcont u(t),\nonumber
\end{align}
where $x(t) \in \R^n$ represents the \textit{state} vector, $y(t)\in\R$ contains the measured \textit{output} and $u(t) \in \R$ represents the control \textit{input} of the system at any time $t \in \R_{\geq 0}$.
$\Acont$, $\Bcont$, $\Ccont$ and $\Dcont$ represent the system, input, output and feedforward matrices of appropriate dimensions. 

The model of Eq.\ \ref{eq:ch7_contsys} is a slight generalization of the model used in Chapter~\ref{chap:parallelisation}, including the feedforward matrix. The embedded implementation we consider is already explained in Section~\ref{sec:ch5_embedded_implementation}.
We assume that the start of sensor-data processing is aligned with the camera frame arrival, i.e., $h$ is an integer multiple of $\fh$. Also, the control computation task and actuation task are delayed, if needed, to guarantee a constant $\tau$ such that $t_s(\taskC^k) = t_s(\taskS^k)+\tau-\actorET_\taskC-\actorET_\taskA$, and $t_s(\taskA^k) = t_s(\taskC^k) + \actorET_\taskC$.
For a non-pipelined implementation (see Fig.~\ref{fig:ch7_ibc_gantt}~(a)), $\tau \le h$, i.e., $t_s(\taskS^{k+1}) \ge t_s(\taskS^k) + \tau$.
For a pipelined implementation (see Fig.~\ref{fig:ch7_ibc_gantt}~(b),~(c)), $\tau > h$, i.e. $t_s(\taskS^{k+1}) < t_s(\taskS^k) + \tau$.
With sensor-to-actuator delay $\tau$ and a zero-order-hold mechanism with sampling period $h \in \R$, $u(t)$ becomes piecewise constant in the intervals $t \in [kh + \tau,\ (k + 1)h + \tau ]$ for $k \in \Ze_{\geq 0}$.

The main challenge here is to compute tight $\tau$ and $h$ for a multiprocessor \gls{ibc} implementation.
Identifying the optimal mapping that guarantees a constant $\tau$ and $h$, considering both pipelining and parallelism together, is non-trivial.

\subsection{\texorpdfstring{\Gls{qoc}}{QoC} metrics for control stability and performance}
\label{sec:ch7_QoC_metrics}
We evaluate the \gls{qoc} of our \gls{ibc} system design choices by considering stability and performance.
Stability margins - gain and phase margins~\cite{12_sunstability} - quantify the control stability and give an analytical basis to compare two different \gls{ibc} system design choices and thus allow us to identify the optimal degree of pipelining and application parallelism.
Once we make a choice, we can further optimise the controller with respect to performance, considering \gls{mse} (explained in Section \ref{sec:ch5_MSE}) and \gls{st}.

The gain margin and phase margin quantify the additional gain and phase lag that makes the system marginally stable.
Systems with greater stability margins can withstand greater changes in system parameters before becoming unstable.
Gain margin and phase margin are computed analytically from the system model.
On the other hand, \gls{mse} and \gls{st} can be analysed only through simulations.

\noindent\textbf{\Acrfull{gm}:} The \gls{gm} is defined as the change in open-loop gain expressed in decibels (dB), required at 180 degrees of phase shift to make the system unstable. 
The \gls{gm} is the difference between the magnitude curve and 0dB at the point corresponding to the frequency that gives us a phase of -180 degrees (the phase cross-over frequency).
 
\noindent\textbf{\Acrfull{pm}:} The \gls{gm} is the change in open-loop phase shift required at unity gain to make a closed-loop system unstable.
The \gls{gm} is the difference in phase between the phase curve and -180 degrees at the point corresponding to the frequency that gives us a gain of 0dB (the gain cross-over frequency).

The control performance quantifies, in essence, how fast the output $y(t)$ reaches the reference $\controlRef$. 
The control performance can be tuned in the cost function for the control gains' design using the state and input weights~\cite{mohamed2019designing}.

\noindent\textbf{\Acrfull{st}:}
The settling time is defined as the time required for the output $y(t)$ to reach and stay within a range of a certain percentage (usually 5\% or 2\%) of the final (reference) value $\controlRef$ forever without external disturbances.