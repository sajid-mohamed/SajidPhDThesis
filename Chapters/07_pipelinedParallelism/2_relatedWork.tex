\section{Related work}
\label{sec:ch7_relatedWork}
This work deals with an effective model-based design flow for multiprocessor \gls{ibc} implementation, considering both pipelining and parallelism together. 
Relevant literature deals with the questions: What are the relevant design approaches for such embedded control problems? What are the techniques for \gls{ibc} system design to deal with long delays in a feedback control loop? What are the relevant \gls{ibc} system modelling and analysis techniques?\\

\noindent\textbf{Design approaches:} An \gls{ibc} system is often designed based on the \emph{separation-of-concerns} principle between the control theory and embedded systems disciplines~\cite{arzen2000introduction,saidi2018future}.
Co-design of control and scheduling is another design paradigm explored in the literature~\cite{cervin2003does}.
The emphasis is on platform-based design methods that take into account platform resource constraints while designing the controller~\cite{arzen2000introduction,goswami2013multirate}.
Contract-based design~\cite{sangiovanni2012taming} is a platform-based design paradigm for cyber-physical systems where the interactions between control theory and embedded design are defined based on contracts.

From the embedded-systems discipline, a system-scenario-based design approach~\cite{gheorghita2009system} is proposed where different behaviours (scenarios) of an application are explicitly considered to avoid over-dimensioning or sub-optimal performance due to worst-case design. 
Identifying, characterising and modelling these scenarios and dealing with the runtime scenario transitions are specific for each application and generally not trivial. 

The \gls{spade} approach for non-pipelined \gls{ibc} systems is proposed in~\cite{mohamed2019designing,mohamed2020scenario} where the concepts of the system-scenario-based design and platform-based design methods for \gls{ibc} are combined into a co-design approach that jointly develops and optimises the image-processing implementation and the controller design.
In this work, we extend the \gls{spade} approach~\cite{mohamed2020scenario} by considering pipelining of the control loop along with parallelism and formalising the \gls{ibc} system modelling.
\\[1ex]
\noindent\textbf{\gls{ibc} system design:}
The main challenge in designing an \gls{ibc} system is to cope with the long sensing delay.
Control engineers tackle a long delay using advanced state estimation~\cite{wang2014multirate}, robust design~\cite{kawamura2012robust}, predictive control~\cite{macesanu2014time}, observer-based~\cite{van2018data}, and multi-rate
sampling~\cite{fujimoto2003visual} methods. 
These methods rely heavily on the system model and are vulnerable to modelling errors with longer delays.
Embedded systems engineers aim to reduce processing delay and period by parallel implementations of the algorithms using heterogeneous multiprocessor platforms having specialised hardware such as GPUs~\cite{agrawal2014gpu,mohamed2020scenario} and FPGAs~\cite{kestur2012emulating}. 
Pipelined control~\cite{krautgartner1998performance,medina2019designing} is another approach targeting homogeneous multiprocessor implementations that reduce the effective sampling period without changing the processing delay.
However, both pipelining and parallelism together for \gls{ibc} systems implementation is not explored in the current literature. 
\\[1ex]
\noindent\textbf{\gls{ibc} system modelling:} Model-based design~\cite{lee1998framework,thiele2000real} approaches focus on designing applications based on abstract models of application and platform such that the implementation is guaranteed to behave with predictable performance. 
Numerous models-of-computation (MoC) are available in literature~\cite{theelen2006scenario,stuijk2010predictable, ChakrabortyKT03, DavisBBL07}. 
Our approach does not depend on a specific MoC.
It needs a MoC that can capture the dynamic behaviours (scenarios) of the application, can analyse timing and has support for platform-aware mapping analysis. We choose \gls{sadf}~\cite{theelen2006scenario} as our MoC as it inherently supports modelling scenarios and has tool support for timing analysis and platform-aware mapping.
The graph transformations we propose in this work are, however, specific for SADF.

Existing literature does not model or explore the impact of both pipelining and parallelism together on the QoC of an \gls{ibc} system.
Inter-frame dependencies, e.g. due to video coding~\cite{li2015lagrangian} or visual tracking~\cite{smeulders2013visual}, and resource sharing between pipes, which are crucial aspects for a practical \gls{ibc} system implementation, are also not explicitly considered in the literature.