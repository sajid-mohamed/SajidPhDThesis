\section{State-of-the-art}
\label{sec:sota}
This thesis explores techniques to cope with the long variable sensing delay by considering application-specific \gls{ibc} system characteristics and exploiting the benefits of multiprocessor platforms. Effectively handling the long variable sensing delay helps to optimise \gls{ibc} system performance while guaranteeing \gls{ibc} system stability.
State-of-the-art methods do not fully exploit the \gls{ibc} system characteristics and advantages of the multiprocessor platforms for optimising the sensing delay. The \gls{ibc} system sensing delay is also variable with a significant degree of variation between the best-case and worst-case delay due to application-specific image-processing workload variations and platform characteristics. A long variable sensing delay degrades system performance and stability. A tight predictable sensing delay is required to optimise the \gls{ibc} system performance and to guarantee the stability of the \gls{ibc} system. Analytical computation of sensing delay is often pessimistic due to image-dependent workload variations or challenging platform timing analysis.

Relevant literature deals with the following questions: 
\begin{itemize}
    \item What are the relevant overall design approaches for such embedded control problems? 
    \item What are the relevant \gls{ibc} system modelling and analysis techniques?
    \item What are the techniques for \gls{ibc} system design to deal with long delays in a feedback control loop?
\end{itemize}

\noindent\textbf{Design approaches:} An \gls{ibc} system is often designed based on the \emph{separation-of-concerns} principle between the control theory and embedded systems disciplines~\cite{arzen2000introduction,saidi2018future}.
Co-design of control and scheduling is another design paradigm explored in the literature~\cite{cervin2003does}.
The emphasis is on platform-based design methods that take into account platform resource constraints while designing the controller~\cite{arzen2000introduction,goswami2013multirate}.
Contract-based design~\cite{sangiovanni2012taming} is a platform-based design paradigm for cyber-physical systems where the interactions between control theory and embedded design are defined based on contracts.

From the embedded-systems discipline, a system-scenario-based design approach~\cite{gheorghita2009system} is proposed where different behaviours (scenarios) of an application are explicitly considered to avoid over-dimensioning or sub-optimal performance due to worst-case design. 
Identifying, characterising and modelling these scenarios and dealing with the runtime scenario transitions are specific for each application and generally not trivial.
In this thesis, we combine scenario-based and platform-based design methods into a coherent \gls{spade} approach for \gls{ibc} system design and implementation.
\\[1ex]
\noindent\textbf{\gls{ibc} system modelling:} Model-based design~\cite{lee1998framework,thiele2000real} approaches focus on designing applications based on abstract models of application and platform such that the implementation is guaranteed to behave with predictable performance. 
Numerous \gls{moc} are available in literature~\cite{theelen2006scenario,stuijk2010predictable, ChakrabortyKT03, DavisBBL07}. 
The approaches in this thesis do not depend on a specific \gls{moc}.
They need a \gls{moc} that can capture the dynamic behaviours (scenarios) of the application, can analyse timing and has support for platform-aware mapping analysis. 
\\[1ex]
\noindent\textbf{\gls{ibc} system design - control engineering perspective:}
The main challenge in designing an \gls{ibc} system is to cope with the long sensing delay.
Control engineers tackle a long delay using advanced state estimation~\cite{wang2014multirate}, robust design~\cite{kawamura2012robust}, predictive control~\cite{macesanu2014time}, observer-based~\cite{van2018data} methods, and multi-rate
sampling~\cite{fujimoto2003visual} methods. 
These methods rely heavily on the system model and are vulnerable to modelling errors with longer delays.
Also, control engineers typically design an \gls{ibc} system considering the worst-case image workload~\cite{saidi2018future} and do not explicitly consider the platform constraints.
In literature, workload variations are typically considered only for sequential \gls{ibc} implementation~\cite{fontantelli2013optimal} and not for parallel and pipelined implementations~\cite{medina2019designing},~\cite{krautgartner1998performance}. 
In~\cite{medina2019implementation}, pipelining is considered along with variable delay.
However, these approaches do not consider system nonlinearities, i.e., the variations in system dynamics, and constraints imposed on the system variables, which can be crucial when considering a practical implementation.  
\\[1ex]
\noindent\textbf{\gls{ibc} system design - embedded systems perspective:}
Embedded systems engineers aim to reduce processing delay and period by parallel implementations of the algorithms using heterogeneous multiprocessor platforms having specialised hardware such as GPUs~\cite{agrawal2014gpu} and FPGAs~\cite{kestur2012emulating}. 
Pipelined control~\cite{krautgartner1998performance,medina2019designing} is another approach targeting homogeneous multiprocessor implementations that reduce the effective sampling period without changing the processing delay.
Inter-frame dependencies, i.e. the data or algorithmic dependencies between consecutive frame processing, e.g. due to video coding~\cite{li2015lagrangian} or visual tracking~\cite{smeulders2013visual}, and resource sharing between pipes, which are crucial aspects for a practical pipelined \gls{ibc} system implementation, are not explicitly considered in the literature.
Additionally, we observe that an integral approach to pipelined parallelism can provide immense benefits to the \gls{ibc} system design and implementation. However, such an integral approach is lacking in literature. 
\\[1ex]
\noindent\textbf{\gls{ibc} system design - approximate computing perspective:}
Approximate computing is another technique that can help reduce the processing delay by trading off computation accuracy for speed. 
Approximations reduce the compute workload across algorithm, architecture and circuit levels. Commonly used algorithmic approximations are computation skipping \cite{comp_skip}, precision scaling \cite{precision_scaling} and replacing error-resilient compute-intensive functions with neural networks \cite{nn_invoke}. A learning approach to design \glspl{isp} for new camera systems is proposed in \cite{learning_isp}. At the architecture level, research efforts have focused on both approximating general-purpose processors\cite{ProACt} as well as domain-specific accelerators \cite{sde}. At the circuit level, research efforts focus on manual design techniques for adders and multipliers \cite{approx_maual}, as well as automated methodologies for designing energy-efficient approximate circuits \cite{sde2}.

Approximation benefits in a camera-based biometric security system, using an iris recognition application, is showcased in \cite{8342029}. An approximate smart camera system is introduced in \cite{araha}, using camera resolution scaling, reducing memory refresh rate and computation skipping. An approximate \gls{isp} pipeline tuned for computer-vision algorithms is designed in \cite{buckler}, by skipping selected \gls{isp} stages. An algorithm-hardware co-designed system is showcased in \cite{euphrates}. 
However, these approximation techniques do not consider a closed-loop feedback system. Approximation decisions in a closed-loop system have quality implications at a later point in time. Optimising a system while accounting for the temporal approximate behaviour is not explored in any of the mentioned work.