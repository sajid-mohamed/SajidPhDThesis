Prior efforts in the approximate computing domain can be broadly classified into compute-centric and data-centric approaches.

\noindent
\textbf{Compute-centric approximations: }Compute-centric efforts are focused on reducing the compute workload across algorithm, architecture and circuit levels. Commonly used algorithmic approximations are computation skipping \cite{comp_skip}, precision scaling \cite{precision_scaling}, and replacing error-resilient compute-intensive functions with neural networks \cite{nn_invoke}. A similar learning approach to design \glspl{isp} for new camera systems is proposed in \cite{learning_isp}. Next, at the architecture level, research efforts have focused on both approximating general-purpose processors\cite{ProACt} as well as domain-specific accelerators \cite{sde}. At the circuit level, research efforts focus on manual design techniques for adders and multipliers \cite{approx_maual}, as well as automated methodologies for designing energy-efficient approximate circuits \cite{sde2}.

\noindent
\textbf{Data-centric approximations: }Data-centric approximations either approximate the memory device that is being accessed or they approximate the value of the data being accessed. Both cases lead to reduced on/off chip data traffic, thereby reducing the required memory bandwidth. Reducing the \gls{dram} refresh rate \cite{dram} and load value speculation \cite{load_speculate} are examples of approximating the memory location. Approximating data values involves storing/accessing data in a compressed format \cite{data_subsetting}. Quality-aware memory controllers for directing memory transactions to different compression schemes are proposed in \cite{mem_control2020}.

\par Both compute and data-centric approaches are focused on approximating individual subsystems. Individual subsystems are usually a part of bigger \gls{ibc} systems. Proper interaction between them is key in ensuring system stability. However, approximating a subsystem might result in undesired behaviour in another, thereby resulting in the failure of the entire \gls{ibc} system.
This is a major downside of approximating each subsystem as a stand-alone entity.

\par To address these limitations, this chapter proposes a holistic full-system analysis approach wherein different subsystems are approximated together in a compute or data-centric manner and the quality implications are evaluated for the full-system rather than individual subsystems. An overview of prior efforts in full-system approximation analysis highlighting the key differences from our work is given below. 

\noindent
\textbf{Full system approximation analysis: }These approaches are targeted at different application domains. For this study, we focus mainly on camera-based systems. Approximation benefits in a camera-based biometric security system, using an iris recognition application, is showcased in \cite{8342029}. An approximate smart camera system is introduced in \cite{araha}, using camera resolution scaling, reducing memory refresh rate and computation skipping. An approximate \gls{isp} pipeline tuned for computer-vision algorithms is designed in \cite{buckler}, by skipping selected \gls{isp} stages. An algorithm-hardware co-designed system is showcased in \cite{euphrates}. It leverages the temporal motion information generated by \glspl{isp} to reduce the compute demands of the perception engine, at the cost of accuracy loss.

\par These research efforts \cite{8342029, araha, buckler, euphrates} lack a closed-loop feedback behaviour. Approximation decisions in a closed-loop system have quality implications at a later point in time. Optimising a system while accounting for the temporal approximate behaviour is not explored in \cite{8342029, araha, buckler, euphrates}, making this a key distinguishable feature of our work. Additionally, looking solely from an approximation perspective, some of the approximation techniques applied in these research efforts can also be applied to our work for additional benefits. For instance, fine-grained computation skipping techniques proposed in \cite{araha, rnr} can be applied in the \gls{isp}, which is an interesting future research direction. Also, leveraging motion information to relax the number of invocations of the perception stage (as shown in \cite{euphrates}) that typically follows the \gls{isp} stage is another interesting research direction.

Table \ref{table_1} summarizes the key contributions of this work stacked up against other state-of-the-art full system approximation approaches.

\begin{table}[t]
	\small
	\renewcommand{\arraystretch}{1.1}
	\caption{{Qualitative comparison with state-of-the-art system-level approximation approaches.}}
	\label{table_1}
	\centering
		\vspace{-5 pt}
	\begin{threeparttable}	
		\setlength\tabcolsep{1.2pt}
		\begin{tabular}{>{\centering\arraybackslash}p{45mm}>{\centering\arraybackslash}p{10mm}>
				{\centering\arraybackslash}p{10mm}>{\centering\arraybackslash}p{10mm}
				>{\centering\arraybackslash}p{10mm}>{\centering\arraybackslash}p{20mm}
				>{\centering\arraybackslash}p{8mm}}
			\hline
			\hline
			\rule{0pt}{8pt}\textbf{Contributions} & \textbf{\cite{8342029}} &\textbf{\cite{araha}} 
			& \textbf{\cite{buckler}} & \textbf{\cite{euphrates}} & \textbf{Our work}\\
			\hline
 			\textbf{compute-centric optimizations} & \checkmark & \checkmark & \checkmark & \checkmark & \checkmark \\
			\hline
            \textbf{data-centric optimizations} &  & \checkmark   & & & \checkmark \\
			\hline
			\textbf{closed-loop approximations *} &  &  & & & \checkmark \\
			\multicolumn{1}{c}{evaluation framework} &  &  & & & \acrshort{sil}, \acrshort{hil}  \\
			platform mappings &  &  & & & \checkmark \\
			environmental scenarios &  &  & & & \checkmark \\
			approximation-aware controller &  &  & & & \checkmark \\
			\hline
			\textbf{\leavevmode{real-system implementation}} & \leavevmode{\checkmark} & \leavevmode{\checkmark} & \leavevmode{$\text{\rlap{$\checkmark$}}\square$} & & \leavevmode{$\text{\rlap{$\checkmark$}}\square$} \\
			\multicolumn{1}{c}{\leavevmode{camera sensor}} & \leavevmode{\checkmark} & \leavevmode{\checkmark} & \leavevmode{$\triangle$} & \leavevmode{$\triangle$} & \leavevmode{$\triangle$}  \\
			\leavevmode{computation} & \leavevmode{\checkmark} & \leavevmode{\checkmark} & \leavevmode{\checkmark} & \leavevmode{$\triangle$} &  \leavevmode{\checkmark} \\
			\leavevmode{controller} &  &  & & &  \leavevmode{\checkmark} \\
			\leavevmode{actuation} &  &  & & &  \leavevmode{$\triangle$} \\
			\hline
		\end{tabular}
		\begin{tablenotes}
			\footnotesize
			\item{$\checkmark$  True, $\text{\rlap{$\checkmark$}}\square$   Partially True, $\triangle$ Modeling or Simulation.}
			\item{\acrshort{sil}: \Acrlong{sil}, \acrshort{hil}: \Acrlong{hil}}
			\item{* \textit{Closed-loop approximations: approximations in closed-loop feedback systems.}}
		\end{tablenotes}
	\end{threeparttable}
	%\vspace{-5 pt}
\end{table}