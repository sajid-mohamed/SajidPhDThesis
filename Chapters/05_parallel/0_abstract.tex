\blfootnote{The content of this chapter is an adaptation of the following two papers:\\
    \bibentry{mohamed2020scenario}.\\
    \bibentry{mohamed2018optimising}.
}
\Gls{ibc} systems are increasingly being used in various domains including autonomous driving. The key challenge in \gls{ibc} is to deal with high computation demand while guaranteeing performance and safety requirements such as stability. 
While modern industrial heterogeneous platforms, such as NVIDIA Drive, offer the necessary compute power, application development on these platforms with performance and safety guarantees is still challenging. Alternative time-predictable platforms are not yet in widespread use.

A typical design flow for \gls{ibc} systems consists of three distinct elements: (i) \emph{mapping} tasks onto platform resources; (ii) \emph{timing analysis}, consisting of \emph{task-level} \gls{wcet} analysis and \emph{application-level} analysis to obtain worst-case performance bounds on aspects such as latency and throughput; (iii) \emph{controller design} using the obtained performance bounds, ensuring performance and safety. While such a three-step design process is modular in nature, it usually leads to over-dimensioned systems with sub-optimal performance, because task- and/or application-level timing bounds are pessimistic.

\begin{sloppypar}
We present a basic scenario- and platform-aware design flow for \gls{ibc} systems that exploits \textit{frequently occurring workload scenarios} to optimize performance. For industrial platforms, where tight task-level \gls{wcet} bounds are difficult to obtain, we moreover propose to use \textit{frequently occurring task execution times} instead of \gls{wcet} estimates to obtain tight application-level temporal bounds.  
During controller design, we then optimize performance and guarantee stability by identifying appropriate \textit{system scenarios} and by designing a \textit{switched controller} that switches between those scenarios. We illustrate our method considering a predictable multiprocessor system-on-chip platform - the \gls{compsoc}. 
We validate the proposed method using \gls{hil} experiments with an industrial heterogeneous multiprocessor platform - NVIDIA Drive PX2 - considering a \gls{lkas}. Both platforms and the \gls{lkas} case study were already introduced in Chapter \ref{chap:intro}. We obtain an improved control performance compared to state-of-the-art \gls{ibc} design.
\end{sloppypar}